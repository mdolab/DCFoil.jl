\documentclass[9pt]{article}
\usepackage{siunitx,gensymb} % gensym gives degree
\usepackage{amsmath}
\usepackage{newtxtext,newtxmath}
\usepackage[numbers]{natbib}
\usepackage{graphicx}
\usepackage{xcolor}
\usepackage[bookmarks=true,bookmarksnumbered=true,colorlinks=false]{hyperref}
\usepackage{cleveref}
\usepackage[margin=1in]{geometry}
\usepackage[compact]{titlesec}

% --- Math ---
\newcommand{\pp}[2]{\frac{\partial #1}{\partial #2}}
\newcommand{\ppt}[2]{\frac{\partial^2 #1}{\partial #2^2}}
\newcommand{\pptt}[2]{\frac{\partial^3 #1}{\partial #2^3}}
\newcommand{\ppttt}[2]{\frac{\partial^4 #1}{\partial #2^4}}
\newcommand{\DD}[2]{\frac{D #1}{D #2}}
\newcommand{\dd}[2]{\frac{d #1}{d #2}}
\newcommand{\ddt}[2]{\frac{d^2 #1}{d #2^2}}
\newcommand{\ddtt}[2]{\frac{d^3 #1}{d #2^3}}
\newcommand{\ddttt}[2]{\frac{d^4 #1}{d #2^4}}
\newcommand{\mbb}[1]{\mathbb{#1}}    % math blackboard
\newcommand{\mbf}[1]{\mathbf{#1}}
\newcommand{\mrm}[1]{\mathrm{#1}}
\newcommand{\mcal}[1]{\mathcal{#1}}  % math calligraphy
\newcommand{\mbfh}[1]{\widehat{\mathbf{#1}}}
\newcommand{\mbfv}[1]{\vec{\mathbf{#1}}}
\newcommand{\half}{\frac{1}{2}}
\newcommand{\third}{\frac{1}{3}}
% --- Equations ---
\newcommand{\be}{\begin{eqnarray}}
\newcommand{\ee}{\end{eqnarray}}
\newcommand{\ben}{\begin{eqnarray*}}
\newcommand{\een}{\end{eqnarray*}}
% --- Numerical methods ---
\newcommand{\dx}{d\mbf{x}}
\newcommand{\Dt}{\Delta t}
\newcommand{\ih}{\hat{i}}
\newcommand{\jh}{\hat{j}}
\newcommand{\kh}{\hat{k}}
\newcommand{\nh}{\hat{n}}
\newcommand{\xh}{\hat{x}}
\newcommand{\yh}{\hat{y}}
\newcommand{\zh}{\hat{z}}
\newcommand{\uj}[1]{u_{j  #1 }}
\newcommand{\ujn}[2]{u_{j  #1 }^{n  #2}}
\newcommand{\BigO}{\mathcal{O}}
\newcommand{\CFL}{{\mbox{CFL}}}
\newcommand{\opt}[1]{#1 (x^{\star})}
% --- Flow ---
\newcommand{\Uinf}{U_{\infty}}
\newcommand{\Vinf}{V_{\infty}}
\newcommand{\Minf}{M_{\infty}}
\newcommand{\bVinf}{\mbf{V}_{\infty}}
\newcommand{\pinf}{p_{\infty}}
\newcommand{\cpmin}{C_{p_{\tn{min}}}}
% --- Other Shorthands ---
\newcommand{\tn}[1]{\textnormal{#1}}
\newcommand{\fw} {forward swept\:}
\newcommand{\un} {unswept\:}
\newcommand{\bw} {backward swept\:}
\newcommand{\scriptth}{\scriptsize \textnormal{th}}
\newcommand{\cfrp}[1] {CFRP #1$^\circ$\:}
\newcommand{\tss}[1]{\textsuperscript{#1}}
\newcommand{\wrt} {with respect to\:}

% --- Spacings ---
\setlength{\headsep}{0.0in}
\setlength{\topmargin}{0.0in}
\setlength{\textheight}{8.8in}
\setlength{\textwidth}{6.8in}
\setlength{\oddsidemargin}{-.15in}
\setlength{\evensidemargin}{-.15in}
\setlength{\unitlength}{1in}

\bibliographystyle{unsrtnat}


\begin{document}

\title{\vspace{-1.5cm} DCFoil Documentation}
\author{Galen W.~Ng}
\maketitle
% ==============================================================================
%                         BEGIN
% ==============================================================================
\section*{Summary}
% 
DCFoil is a program for the dynamic analysis and design optimization of composite hydrofoils.


\clearpage
\tableofcontents
\clearpage
%------------------------------------------------------------------------------
\section{Coordinate system}
% 

%------------------------------------------------------------------------------
\section{Discretization}
\subsection{Structural beam model}
\subsection{Hydrodynamic loads}
% 
The lifting line model derives from~\citet[Ch. XI]{Glauert1983a} and works for arbitrary chord.
Specifically, we are after sectional lift slopes ($\textstyle c_{\ell_\alpha} = dc_\ell/d\alpha = a_0$).
We assume
\begin{itemize}
    \item the chord is small compared to the span,
    \item the wing is symmetric about the centerline,
    \item span is straight and orthogonal to the freestream
    \item trailing vortices are shed from the trailing edge and align with the freestream
\end{itemize}
The wing is represented by superimposing ``horseshoe'' systems of vortex lines (analagous to a wire with electrical current).
This is because the circulation across a wing is not constant.
The free vortex system is a sheet of trailing vortices springing from the trailing edge.
The induced velocity of an element of the line ($ds$) at point $P$ from one vortex line of constant strength $\Gamma$ is
\be
dq = \frac{\Gamma}{4 \pi r^2} \sin(\theta) ds
\ee
but in practice, one would solve this is an integral over the entire vortex line, so we will build up to the full wing.

% We know the forces to be
% \begin{align}
%      & L = \int_{-s/2}^{s/2} \rho \Uinf \Gamma(y) dy          \\
%      & D_i = \int_{-s/2}^{s/2} \rho \Uinf w(y)  \Gamma(y) dy
% \end{align}
% where $s$ is total span.
To begin solution, we first assume the circulation is the Fourier series\footnote{\citet{Kerwin2010} use $\tilde{y}$ as $\theta$}
\be
\Gamma(y) = 2 \Uinf s \sum_{n=1}^{\infty} a_n \sin \left( n \theta \right)
\quad \tn{where }
y = -\frac{s}{2}\cos(\theta)
\quad \tn{and }
\theta \in \left[-\frac{\pi}{2},\frac{\pi}{2}\right]
.
\ee
The difficulty is now determining the Fourier coefficients $a_n$ so we need some relations for $\Gamma(y)$ to solve it.

One relation is the equation for the normal induced velocity (downwash velocity) at a point along the span
\be
\label{eq:downwash}
w(y) = \frac{1}{4\pi}
\int_{-s/2}^{s/2} \frac{\dd{\Gamma}{\eta} }{y - \eta} d\eta
=
\boxed{
    -\Uinf \sum_{n=1}^{\infty} \frac{n a_n \sin \left( n \theta \right)}{\sin(\theta)}
}
\ee
where $\eta$ is the spanwise coordinate and $s$ is total span.
We skipped a few steps in the derivation~\cite[Sec.~3.7]{Kerwin2010}.

The second relation is from sectional lift as a function of circulation.
Recall that the circulation at a section (derived from Kutta-Joukowski lift theorem) is
\be
\label{eq:circulation}
\Gamma(y) = \half c_\ell c \Uinf
=\half a_0 c \left(\Uinf \alpha - w(y)\right)
\ee
where we made use of $c_\ell = a_0 \alpha_{\tn{eff}} = a_0 \left(\alpha - w/\Uinf\right)$.
After substitution of the Fourier series form and combining Equations~\eqref{eq:downwash} and~\eqref{eq:circulation}, we end up with
\be
\label{eq:circulation2}
\sum_{n=1}^{\infty} a_n \sin(n\theta) \left(n\mu + \sin(\theta)\right)
=
\mu \alpha \sin(\theta)
\quad \tn{where }\mu(\theta) = \frac{a_0 c(\theta)}{ 4 s}
\ee

Here's the digestion of the Julia code which does the numerical solution of Equation~\eqref{eq:circulation2} but symmetrically about the centerline.
\begin{align*}
     & \tilde{y} = \left[
        0,\frac{\pi}{2}
        \right] \quad \tn{of size nNodes}
    \\
     & \mbf{n} = \left[1:2:2\times\tn{nNodes}\right]
    \\
     & \mbf{c} = c  \sin(\tilde{y}) \quad \tn{(parametrized vector leading to elliptical planform)}
    \\
     & \mbf{b} = \frac{\pi}{4} \frac{\mbf{c}}{s/2} \alpha \sin(\tilde{y}) \quad \tn{(RHS of Equation~\eqref{eq:circulation2} in vector form)}
    \\
     & \tilde{y}n = \tilde{y} \otimes \mbf{n} \quad \tn{(outer product)}
    \\
     & \mbf{A_0} = \begin{bmatrix}
                       |               & |                        \\
                       \sin(\tilde{y}) & \sin(\tilde{y}) & \cdots \\
                       |               & |
                   \end{bmatrix} \quad \tn{(square matrix of $\sin(\tilde{y})$)}
    \\
     & \mbf{A_1} = \frac{\pi}{4} \frac{\mbf{c}}{s/2} \otimes \mbf{n} \quad \tn{(outer product representing $n\mu$ on LHS)}
    \\
     & \mbf{A} = \sin(\tilde{y} n ) \odot \left( \mbf{A_0}+\mbf{A_1} \right)                                                                  \\
     & \mbf{A}\mbf{x} = \mbf{b} \quad \tn{(solve linear system for $\mbf{x} = a_n$)}                                                          \\
     & \Gamma(y) = 4 \Uinf s/2 \left(\underbrace{\sin(\tilde{y}n)\mbf{x} }_{\tn{mat-vec product}} \right)                                     \\
     & c_\ell = \frac{2\Gamma(y)}{\Uinf {c}}                                                                                                  \\
     & c_{\ell_{\alpha}} = \frac{c_\ell}{\alpha}
\end{align*}
%------------------------------------------------------------------------------
\section{Static solution}

%------------------------------------------------------------------------------
\section{Forced vibration solution}

%------------------------------------------------------------------------------
\section{Flutter solution}

\bibliography{./gng-link,./mdolab-link}

\end{document}